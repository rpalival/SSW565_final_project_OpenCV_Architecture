\chapter{User Stories\\
\small{\textit{-- Nikhil Kumar G, Raj Palival}}
\index{Chapter!User Stories}
\index{User Stories}
\label{Chapter::User Stories}}

\section{What is a User Story ? \label{Section::UserStoriesIntroduction}}
A user story is a small, self-contained unit of development work designed to accomplish a specific goal within a product. A user story is usually written from the user’s perspective and follows the format: “As [a user persona], I want [to perform this action] so that [I can accomplish this goal].”\newline
\newline
User stories:
\begin{itemize}
\item Are easy for anyone to understand
\item Represent bite-sized deliverables that can fit in sprints, whereas not all full features can. \index{bite-sized deliverables}
\item Help the team focus on real people, rather than abstract features
\item Build momentum by giving development teams a feeling of progress
\end{itemize}
\section{User Stories\label{Section::UserStories}}
\begin{enumerate}
    \item As a developer, I want to use OpenCV to easily integrate computer vision capabilities into my software applications, so that I can create more advanced and intelligent applications that can analyze and interpret visual data.
    \item As a contributor, I want to contribute to the development of OpenCV by submitting bug reports, fixing issues, and adding new features, so that I can help improve the quality and functionality of the software for the benefit of the community.
    \item As the OpenCV foundation, we want to promote the use and development of OpenCV by providing resources, support, and guidance to developers, researchers, and industry partners, so that we can advance the field of computer vision and make it more accessible to everyone.
    \item As a member of the open source community, I want to collaborate with other developers and contributors to improve OpenCV, share my knowledge and expertise, and help make computer vision more accessible and useful to everyone.
    \item As an end-user, I want to use OpenCV to solve real-world problems, such as object recognition, face detection, and image processing, so that I can improve my productivity, efficiency, and quality of life.
    \item As a researcher, I want to use OpenCV to conduct experiments, analyze data, and develop new algorithms and techniques in the field of computer vision, so that I can advance the state of the art and contribute to the scientific community \cite{888718}.
    \item As an academic institution, we want to use OpenCV to teach computer vision concepts and techniques to our students, conduct research, and collaborate with other institutions and industry partners, so that we can prepare our students for careers in this rapidly growing field \cite{9103956}.
    \item As an industry partner, we want to use OpenCV to develop innovative products and services that leverage computer vision technology, so that we can improve our competitiveness, create new business opportunities, and provide value to our customers \cite{8097324}.
    \item As a government agency, we want to use OpenCV to develop solutions that can help us address various challenges, such as public safety, transportation, and environmental monitoring, so that we can improve the quality of life for our citizens and promote economic growth.
\end{enumerate}