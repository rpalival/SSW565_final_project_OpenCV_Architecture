\chapter{Business Requirements\\
\small{\textit{-- Nikhil Kumar G, Raj Palival}}
\index{Chapter!Business Requirements}
\index{Business Requirements}
\label{Chapter::Business Requirements}}

\section{Introduction \label{Section::BusinessRequirementIntro}} \index{Business Requirements}
Business requirements outline a project's needs and prerequisites for success while taking the target audience into account. It explains the rationale behind developing a certain project, who will use it, what advantages users will experience, and how the project's success will be measured. Business requirements do not specify how the project is to be developed \cite{businessRequirement}.
\subsection{What does a business requirement include ?\label{SubSection::BRInclusions}}
\begin{itemize}
    \item Key objectives and identification of a problem.
    \item Benefits of the proposed solution.
    \item Project scope.
    \item Rules, regulations, and policies.
    \item Key features of the project.
    \item Performance and security features.
    \item Metrics to measure the success of the project.
\end{itemize}
\subsection{What does a business requirement not include ?\label{SubSection::BRExclusions}}
\begin{itemize}
    \item Details of the functional requirements of the system.
    \item Details of the implementation of functional or performance requirements.
    \item Details of how to implement policies and regulations.
\end{itemize}
\newpage
\section{Business Requirements\label{Section::BusinessRequirements}}
\begin{enumerate}
    \item Open source: Develop and maintain an open source computer vision and machine learning software library that provides a common infrastructure for computer vision applications, which will allow developers to contribute to the project and make it better. This will also make OpenCV more accessible to a wider range of developers.
    \item Real-time performance: OpenCV must be able to process images and videos in real time, which is essential for many applications such as facial recognition, object detection, and video surveillance.
    \item Documentation: OpenCV must have comprehensive documentation, which will help developers learn how to use the library. This will make it easier for developers to get started with OpenCV and build their applications.
    \item Image Processing: OpenCV should be able to process images of various formats, including JPEG, PNG, and BMP. It should be able to perform basic image processing operations such as resizing, cropping, and rotating images. Additionally, it should be able to apply filters such as blur, sharpen, and edge detection to images.
    \item Hardware acceleration: OpenCV must support hardware acceleration, such as CUDA and OpenCL, to improve performance. This will allow developers to take advantage of the latest hardware to speed up their applications. \index{Hardware acceleration} \index{CUDA} \index{OpenCL} \index{video surveillance}
    \item Support: OpenCV should have a strong support system that provides timely and effective assistance to users. It should have a large and active community forum where users can ask questions and share their experiences. It should also have a dedicated support team that can help users troubleshoot issues and provide solutions to problems.
    \item Cross-platform compatibility: Develop and maintain a computer vision and machine learning software library that can be used on various platforms, including Windows, Linux, Android, and Mac OS. This will allow developers to use OpenCV on a variety of devices and systems. \index{Cross-platform} \index{Windows} \index{Linux} \index{Android} \index{Mac OS}
    \item Lowering costs: Enable businesses to utilize and modify the code for their specific needs, without incurring high licensing costs.
    \item Multi-Language Support: Ensure compatibility with popular programming languages, including C++, Python, Java, and MATLAB. \index{C++} \index{Python} \index{Java} \index{MATLAB} \index{Customization} \index{filters}
    \item Customization: OpenCV should be customizable to meet the specific needs of the business. It should allow users to create custom filters, algorithms, and models. It should also allow users to modify the user interface to suit their preferences and workflows.
\end{enumerate}